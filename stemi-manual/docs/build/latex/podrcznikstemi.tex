%% Generated by Sphinx.
\def\sphinxdocclass{report}
\documentclass[letterpaper,10pt,polish]{sphinxmanual}
\ifdefined\pdfpxdimen
   \let\sphinxpxdimen\pdfpxdimen\else\newdimen\sphinxpxdimen
\fi \sphinxpxdimen=.75bp\relax
\ifdefined\pdfimageresolution
    \pdfimageresolution= \numexpr \dimexpr1in\relax/\sphinxpxdimen\relax
\fi
%% let collapsible pdf bookmarks panel have high depth per default
\PassOptionsToPackage{bookmarksdepth=5}{hyperref}

\PassOptionsToPackage{booktabs}{sphinx}
\PassOptionsToPackage{colorrows}{sphinx}

\PassOptionsToPackage{warn}{textcomp}
\usepackage[utf8]{inputenc}
\ifdefined\DeclareUnicodeCharacter
% support both utf8 and utf8x syntaxes
  \ifdefined\DeclareUnicodeCharacterAsOptional
    \def\sphinxDUC#1{\DeclareUnicodeCharacter{"#1}}
  \else
    \let\sphinxDUC\DeclareUnicodeCharacter
  \fi
  \sphinxDUC{00A0}{\nobreakspace}
  \sphinxDUC{2500}{\sphinxunichar{2500}}
  \sphinxDUC{2502}{\sphinxunichar{2502}}
  \sphinxDUC{2514}{\sphinxunichar{2514}}
  \sphinxDUC{251C}{\sphinxunichar{251C}}
  \sphinxDUC{2572}{\textbackslash}
\fi
\usepackage{cmap}
\usepackage[T1]{fontenc}
\usepackage{amsmath,amssymb,amstext}
\usepackage{babel}



\usepackage{tgtermes}
\usepackage{tgheros}
\renewcommand{\ttdefault}{txtt}



\usepackage[Sonny]{fncychap}
\ChNameVar{\Large\normalfont\sffamily}
\ChTitleVar{\Large\normalfont\sffamily}
\usepackage{sphinx}

\fvset{fontsize=auto}
\usepackage{geometry}


% Include hyperref last.
\usepackage{hyperref}
% Fix anchor placement for figures with captions.
\usepackage{hypcap}% it must be loaded after hyperref.
% Set up styles of URL: it should be placed after hyperref.
\urlstyle{same}

\addto\captionspolish{\renewcommand{\contentsname}{Contents:}}

\usepackage{sphinxmessages}
\setcounter{tocdepth}{2}



\title{Podręcznik SteMi}
\date{29 cze 2023}
\release{0.1}
\author{Krzysztof Stenografow Smirnow}
\newcommand{\sphinxlogo}{\vbox{}}
\renewcommand{\releasename}{Wydanie}
\makeindex
\begin{document}

\ifdefined\shorthandoff
  \ifnum\catcode`\=\string=\active\shorthandoff{=}\fi
  \ifnum\catcode`\"=\active\shorthandoff{"}\fi
\fi

\pagestyle{empty}
\sphinxmaketitle
\pagestyle{plain}
\sphinxtableofcontents
\pagestyle{normal}
\phantomsection\label{\detokenize{index::doc}}


\sphinxstepscope


\chapter{Cość}
\label{\detokenize{01:cosc}}\label{\detokenize{01::doc}}

\section{My nifty title}
\label{\detokenize{01:my-nifty-title}}
\sphinxAtStartPar
Some \sphinxstylestrong{text}! I skreślone

\begin{sphinxadmonition}{note}{Uwaga!!!}

\sphinxAtStartPar
Here’s my admonition content.$^{\text{1}}$
\end{sphinxadmonition}

\sphinxAtStartPar
\sphinxincludegraphics{{wtedy}.jpg}

\sphinxAtStartPar
\sphinxincludegraphics{{wtedy}.png}

\begin{sphinxadmonition}{tip}{Wskazówka:}
\sphinxAtStartPar
Let’s give readers a helpful hint!
\end{sphinxadmonition}

\sphinxAtStartPar
W 2015 roku opublikowano także wyniki badań lipidowych z naczyń ceramicznych wykorzystywanych przez społeczności zamieszkujące w neolicie zachodnią i środkową Europę oraz Bliski Wschód. Była to pierwsza publikacja przedstawiająca specjalistyczne biomarkery świadczące o przechowywaniu miodu w naczyniach ceramicznych. O ile miód złożony jest w większości z całkowicie rozpuszczalnych w wodzie cukrów (fruktoza i glukoza), o tyle inne produkty pszczele, w tym wosk, mogą pozostawić po sobie charakterystyczne ślady takie jak biomarkery lub n\sphinxhyphen{}alkany specyficzne przykładowo dla wosku pszczelego. Jak słusznie założyli badacze, w przeszłości pszczelarze prawdopodobnie nie byli na tyle wyspecjalizowani, aby oddzielać dokładnie i skrupulatnie miód od innych produktów pszczelich.

\sphinxstepscope


\chapter{Durga}
\label{\detokenize{02/02:durga}}\label{\detokenize{02/02::doc}}
\begin{sphinxadmonition}{note}{Uwaga!!!}

\sphinxAtStartPar
Here’s my admonition content.$^{\text{1}}$
\end{sphinxadmonition}

\sphinxAtStartPar
\sphinxincludegraphics{{wtedy}.jpg}

\sphinxAtStartPar
\sphinxincludegraphics{{wtedy}.png}

\sphinxAtStartPar
Miód był \sphinxincludegraphics[width=50\sphinxpxdimen]{{kiedy}.jpg} wykorzystywany w celach kulinarnych co najmniej od mezolitu.

\sphinxAtStartPar
Jednakże inne \sphinxincludegraphics[width=50\sphinxpxdimen]{{kiedy}.jpg} produkty pszczele człowiek stosował do pielęgnacji, w sztuce oraz jako surowiec technologiczny prawdopodobnie od paleolitu. Najbardziej problematyczne jest to, że wszystkie te produkty nie pozostawiają po sobie żadnych śladów widocznych gołym okiem ani śladów uchwytnych mikroskopowo. Dopiero przy zastosowaniu metod lipidowych można uchwycić to, że pozostałości wyrobów pszczelich były wykorzystywane przez ludzi. Jeśli przechowywano je w naczyniach, które zachowały się do naszych czasów.

\begin{sphinxadmonition}{caution}{Ostrzeżenie:}
\sphinxAtStartPar
Uwaga!
\end{sphinxadmonition}


\begin{savenotes}\sphinxattablestart
\sphinxthistablewithglobalstyle
\centering
\begin{tabulary}{\linewidth}[t]{TTT}
\sphinxtoprule
\sphinxstyletheadfamily 
\sphinxAtStartPar
Tables
&\sphinxstyletheadfamily 
\sphinxAtStartPar
Are
&\sphinxstyletheadfamily 
\sphinxAtStartPar
Cool
\\
\sphinxmidrule
\sphinxtableatstartofbodyhook
\sphinxAtStartPar
col 1 is
&
\sphinxAtStartPar
left\sphinxhyphen{}aligned
&
\sphinxAtStartPar
\$1600
\\
\sphinxhline
\sphinxAtStartPar
col 2 is
&
\sphinxAtStartPar
centered
&
\sphinxAtStartPar
\$12
\\
\sphinxhline
\sphinxAtStartPar
col 3 is
&
\sphinxAtStartPar
right\sphinxhyphen{}aligned
&
\sphinxAtStartPar
\$1
\\
\sphinxbottomrule
\end{tabulary}
\sphinxtableafterendhook\par
\sphinxattableend\end{savenotes}

\begin{sphinxadmonition}{danger}{Niebezpieczeństwo:}
\sphinxAtStartPar
Nieprzespiecznie
\end{sphinxadmonition}

\sphinxAtStartPar
\sphinxincludegraphics[width=50\sphinxpxdimen]{{kiedy}.jpg} kiedy | \sphinxincludegraphics[width=50\sphinxpxdimen]{{jezeli}.jpg} jeżeli

\sphinxAtStartPar
\sphinxincludegraphics[width=50\sphinxpxdimen]{{kiedy}.jpg} kiedy \\
\sphinxincludegraphics[width=50\sphinxpxdimen]{{jezeli}.jpg} jeżeli

\sphinxAtStartPar
\sphinxincludegraphics[width=50\sphinxpxdimen]{{jezeli}.jpg} jeżeli
\sphinxincludegraphics[width=50\sphinxpxdimen]{{bedzie}.jpg} będzie
\sphinxincludegraphics[width=50\sphinxpxdimen]{{nie}.jpg} nie
\sphinxincludegraphics[width=50\sphinxpxdimen]{{kiedy}.jpg} kiedy
\sphinxincludegraphics[width=50\sphinxpxdimen]{{raz}.jpg} raz
\sphinxincludegraphics[width=50\sphinxpxdimen]{{wtedy1}.jpg} wtedy

\sphinxAtStartPar
\sphinxincludegraphics{{jezeli}.jpg} jeżeli
\sphinxincludegraphics{{bedzie}.jpg} będzie
\sphinxincludegraphics{{nie}.jpg} nie
\sphinxincludegraphics{{kiedy}.jpg} kiedy
\sphinxincludegraphics{{raz}.jpg} raz
\sphinxincludegraphics{{wtedy1}.jpg} wtedy

\sphinxAtStartPar
\sphinxincludegraphics{{jezeli}.jpg}
\sphinxincludegraphics{{bedzie}.jpg}
\sphinxincludegraphics{{nie}.jpg}
\sphinxincludegraphics{{kiedy}.jpg}
\sphinxincludegraphics{{raz}.jpg}
\sphinxincludegraphics{{wtedy1}.jpg}

\begin{figure}[htbp]
\centering
\capstart

\noindent\sphinxincludegraphics[width=50\sphinxpxdimen]{{jezeli}.jpg}
\caption{jeżeli}\label{\detokenize{02/02:id1}}\end{figure}


\section{Trzeci poziom}
\label{\detokenize{02/02:trzeci-poziom}}
\begin{sphinxadmonition}{error}{Błąd:}
\sphinxAtStartPar
Błąd
\end{sphinxadmonition}


\subsection{Czwarty poziom}
\label{\detokenize{02/02:czwarty-poziom}}
\begin{sphinxadmonition}{important}{Ważne:}
\sphinxAtStartPar
Importante
\end{sphinxadmonition}

\sphinxAtStartPar
\sphinxincludegraphics[width=100\sphinxpxdimen]{{hha}.jpg}

\begin{sphinxadmonition}{note}{Informacja:}
\sphinxAtStartPar
Note
\end{sphinxadmonition}


\begin{sphinxseealso}{Zobacz także:}

\sphinxAtStartPar
Patrz też seealso


\end{sphinxseealso}


\begin{sphinxadmonition}{warning}{Ostrzeżenie:}
\sphinxAtStartPar
Warning
\end{sphinxadmonition}

\sphinxAtStartPar
\sphinxincludegraphics{{hha}.jpg}

\sphinxAtStartPar
Na podstawie tego typu badań jesteśmy w stanie uchwycić występowanie materiału dotychczas uznawanego za nieuchwytny archeologicznie. Dodatkowo większość przedmiotów związanych z tradycyjnym pszczelarstwem w Europie także wykonywano z surowców organicznych, które rzadko zachowują się w kontekście archeologicznym. Należą do nich choćby ule, w tym tradycyjne dla niektórych rejonów zachodniej Europy skeps, czyli plecione ule, wykonywane najczęściej z wikliny.


\chapter{Indeksy i tabele}
\label{\detokenize{index:indeksy-i-tabele}}\begin{itemize}
\item {} 
\sphinxAtStartPar
\DUrole{xref,std,std-ref}{genindex}

\item {} 
\sphinxAtStartPar
\DUrole{xref,std,std-ref}{modindex}

\item {} 
\sphinxAtStartPar
\DUrole{xref,std,std-ref}{search}

\end{itemize}


\chapter{Odsyłacze robocze}
\label{\detokenize{index:odsylacze-robocze}}\begin{itemize}
\item {} 
\sphinxAtStartPar
\sphinxhref{https://bashtage.github.io/sphinx-material/rst-cheatsheet/rst-cheatsheet.html}{RST Cheatsheet}

\item {} 
\sphinxAtStartPar
\sphinxhref{https://myst-parser.readthedocs.io/en/latest/index.html}{MYST docs}

\end{itemize}



\renewcommand{\indexname}{Indeks}
\printindex
\end{document}