%% Generated by Sphinx.
\def\sphinxdocclass{report}
\documentclass[letterpaper,10pt,polish]{sphinxmanual}
\ifdefined\pdfpxdimen
   \let\sphinxpxdimen\pdfpxdimen\else\newdimen\sphinxpxdimen
\fi \sphinxpxdimen=.75bp\relax
\ifdefined\pdfimageresolution
    \pdfimageresolution= \numexpr \dimexpr1in\relax/\sphinxpxdimen\relax
\fi
%% let collapsible pdf bookmarks panel have high depth per default
\PassOptionsToPackage{bookmarksdepth=5}{hyperref}

\PassOptionsToPackage{booktabs}{sphinx}
\PassOptionsToPackage{colorrows}{sphinx}

\PassOptionsToPackage{warn}{textcomp}
\usepackage[utf8]{inputenc}
\ifdefined\DeclareUnicodeCharacter
% support both utf8 and utf8x syntaxes
  \ifdefined\DeclareUnicodeCharacterAsOptional
    \def\sphinxDUC#1{\DeclareUnicodeCharacter{"#1}}
  \else
    \let\sphinxDUC\DeclareUnicodeCharacter
  \fi
  \sphinxDUC{00A0}{\nobreakspace}
  \sphinxDUC{2500}{\sphinxunichar{2500}}
  \sphinxDUC{2502}{\sphinxunichar{2502}}
  \sphinxDUC{2514}{\sphinxunichar{2514}}
  \sphinxDUC{251C}{\sphinxunichar{251C}}
  \sphinxDUC{2572}{\textbackslash}
\fi
\usepackage{cmap}
\usepackage[T1]{fontenc}
\usepackage{amsmath,amssymb,amstext}
\usepackage{babel}



\usepackage{tgtermes}
\usepackage{tgheros}
\renewcommand{\ttdefault}{txtt}



\usepackage[Sonny]{fncychap}
\ChNameVar{\Large\normalfont\sffamily}
\ChTitleVar{\Large\normalfont\sffamily}
\usepackage{sphinx}

\fvset{fontsize=auto}
\usepackage{geometry}


% Include hyperref last.
\usepackage{hyperref}
% Fix anchor placement for figures with captions.
\usepackage{hypcap}% it must be loaded after hyperref.
% Set up styles of URL: it should be placed after hyperref.
\urlstyle{same}

\addto\captionspolish{\renewcommand{\contentsname}{Spis treści:}}

\usepackage{sphinxmessages}
\setcounter{tocdepth}{2}



\title{Podręcznik SteMi}
\date{12 lip 2023}
\release{0.1}
\author{Krzysztof Stenografow Smirnow}
\newcommand{\sphinxlogo}{\vbox{}}
\renewcommand{\releasename}{Wydanie}
\makeindex
\begin{document}

\ifdefined\shorthandoff
  \ifnum\catcode`\=\string=\active\shorthandoff{=}\fi
  \ifnum\catcode`\"=\active\shorthandoff{"}\fi
\fi

\pagestyle{empty}
\sphinxmaketitle
\pagestyle{plain}
\sphinxtableofcontents
\pagestyle{normal}
\phantomsection\label{\detokenize{index::doc}}


\sphinxstepscope


\chapter{Alfabet SteMi}
\label{\detokenize{alfabet-stemi:alfabet-stemi}}\label{\detokenize{alfabet-stemi::doc}}
\sphinxAtStartPar
System stenograficzny Ste\sphinxstyleemphasis{Mi} wywodzi się z alfabetów, czyli piszemy w nim literkami. Jednak należy pamiętać, że obowiązuje podejście fonetyczne, a właściwie*\sphinxhref{https://pl.wikipedia.org/wiki/Fonologia}{fonologiczne}*, tj. dane znaki oddają \sphinxstylestrong{brzmienie} i nie obowiązują piszącego znane ze szkoły zasady ortografii.

\begin{sphinxadmonition}{important}{Ważne:}
\sphinxAtStartPar
Brak ścisłych reguł ortograficznych wynika z faktu, że nie zwracamy uwagi na pochodzenie i związki pomiędzy wyrazami, które stanowią podstawę polskiej ortografii. Najważniejsze jest oddanie słowa w jego brzmieniu, LUB zastosowanie niewątpliwego ZNACZNIKA.
\end{sphinxadmonition}

\sphinxAtStartPar
Brak ortografii nie oznacza, że nie obowiązują żadne reguły. Ze względu na priorytety w systemie stenograficznym, lepiej ich sens oddaje słowo \sphinxstyleemphasis{metody}. W rzeczywistości im więcej metod skracania, tym szybciej można pisać.

\begin{sphinxadmonition}{note}{Informacja:}
\sphinxAtStartPar
Ste\sphinxstyleemphasis{Mi} nie jest systemem sztywnym. Służy przede wszystkim wygodzie piszącego. Zatem to od niego zależy, które reguły/metody znajdą u niego zastosowanie. Piszący może też tworzyć własne reguły/metody, jeżeli dostrzeże taką potrzebę. A jeżeli uzna, że mogą one być przydatne ogółowi użytkowników systemu, najlepiej, gdyby wysłał ich opracowanie na adres \sphinxstyleemphasis{\sphinxhref{mailto:flamenco108@stenografia.pl}{flamenco108@stenografia.pl}}.
\end{sphinxadmonition}


\section{Samogłoski}
\label{\detokenize{alfabet-stemi:samogloski}}
\sphinxAtStartPar
Samogłoski SteMi mogą występować w różnych wariantach, które mają swoje zastosowanie. Zostanie to omówione dalej.

\begin{figure}[htbp]
\centering
\capstart

\noindent\sphinxincludegraphics{{samogloski-litery}.png}
\caption{Od lewej: A Y E Ę I O Ą U \\
Litery \sphinxstyleemphasis{I}, \sphinxstyleemphasis{O}, \sphinxstyleemphasis{Ą}, \sphinxstyleemphasis{U} występują w różnych wariantach.}\label{\detokenize{alfabet-stemi:id1}}\end{figure}

\begin{figure}[htbp]
\centering
\capstart

\noindent\sphinxincludegraphics{{samogloski}.png}
\caption{Znaki samogłoskowe z czcionki SteMi: \\
A , Y , E , Ę , O , Ą , U , I(j)}\label{\detokenize{alfabet-stemi:id2}}\end{figure}

\begin{sphinxadmonition}{important}{Ważne:}
\sphinxAtStartPar
Samogłoski piszemy \sphinxstylestrong{generalnie} \sphinxstyleemphasis{wznosząco}, a spółgłoski piszemy \sphinxstylestrong{generalnie} \sphinxstyleemphasis{opadająco}.
\end{sphinxadmonition}

\begin{figure}[htbp]
\centering
\capstart

\noindent\sphinxincludegraphics{{samogloski_old2}.png}
\caption{Samogłoski pisane są generalnie wznosząco: \\
A Y ,  E Ę ,  U ,  O ,  Ą ,  I(j)}\label{\detokenize{alfabet-stemi:id3}}\end{figure}


\section{Spółgłoski}
\label{\detokenize{alfabet-stemi:spolgloski}}
\begin{figure}[htbp]
\centering
\capstart

\noindent\sphinxincludegraphics{{spolgloski-litery}.png}
\caption{K ki/k’  / G gi/g’ / P pi/p’ / B bi/b’ / H hi/h’ / M  mi/m’ / F  fi/f’ / W wi/w’  / N ni/ń \\
S Z L / C DZ Ł / CZ DŻ / SZ Ż/RZ / Ś Ź / Ć DŹ / T D / R}\label{\detokenize{alfabet-stemi:id4}}\end{figure}

\sphinxAtStartPar
Znaki spółgłoskowe zostały skomponowane w taki sposób, aby dźwięki podobne pisane były podobnie. Oczywiście wszystkie te reguły należy traktować z przymrużeniem oka.

\begin{figure}[htbp]
\centering
\capstart

\noindent\sphinxincludegraphics{{spolgloski}.png}
\caption{Znaki spółgłoskowe z czcionki SteMi:
K/G P/B T/D H/M/N F/W S/Z/L \\
C/DZ/Ł SZ/Ż CZ/DŻ Ś/Ź Ć/DŹ R}\label{\detokenize{alfabet-stemi:id5}}\end{figure}

\begin{figure}[htbp]
\centering
\capstart

\noindent\sphinxincludegraphics{{spolgloski_old2}.png}
\caption{Spółgłoski piszemy generalnie opadająco: \\
K G ,  P B ,  H M ,  F W ,  S Z L ,  R \\
S D ,  Ś Ź ,  Ć DŹ,  SZ Ż ,  CZ DŻ ,  C DZ Ł}\label{\detokenize{alfabet-stemi:id6}}\end{figure}


\section{Litery przegląd}
\label{\detokenize{alfabet-stemi:litery-przeglad}}

\begin{savenotes}\sphinxattablestart
\sphinxthistablewithglobalstyle
\centering
\begin{tabulary}{\linewidth}[t]{TTTTTTT}
\sphinxtoprule

\sphinxAtStartPar

&
\sphinxAtStartPar

&
\sphinxAtStartPar

&
\sphinxAtStartPar

&
\sphinxAtStartPar

&
\sphinxAtStartPar

&
\sphinxAtStartPar

\\
\sphinxmidrule
\sphinxtableatstartofbodyhook
\sphinxAtStartPar
a
&
\sphinxAtStartPar
y
&
\sphinxAtStartPar
e
&
\sphinxAtStartPar
ę
&
\sphinxAtStartPar
i/j
&
\sphinxAtStartPar
o
&
\sphinxAtStartPar
u
\\
\sphinxhline
\sphinxAtStartPar
\sphinxincludegraphics[height=80\sphinxpxdimen]{{a}.png}
&
\sphinxAtStartPar
\sphinxincludegraphics[height=80\sphinxpxdimen]{{y}.png}
&
\sphinxAtStartPar
\sphinxincludegraphics[height=80\sphinxpxdimen]{{e}.png}
&
\sphinxAtStartPar
\sphinxincludegraphics[height=80\sphinxpxdimen]{{e_}.png}
&
\sphinxAtStartPar
\sphinxincludegraphics[height=80\sphinxpxdimen]{{i}.png}
&
\sphinxAtStartPar
\sphinxincludegraphics[height=80\sphinxpxdimen]{{o}.png}
&
\sphinxAtStartPar
\sphinxincludegraphics[height=80\sphinxpxdimen]{{u}.png}
\\
\sphinxhline
\sphinxAtStartPar

&
\sphinxAtStartPar

&
\sphinxAtStartPar

&
\sphinxAtStartPar

&
\sphinxAtStartPar

&
\sphinxAtStartPar

&
\sphinxAtStartPar

\\
\sphinxhline
\sphinxAtStartPar
k
&
\sphinxAtStartPar
ki
&
\sphinxAtStartPar
g
&
\sphinxAtStartPar
gi
&
\sphinxAtStartPar
p
&
\sphinxAtStartPar
pi
&
\sphinxAtStartPar
r
\\
\sphinxhline
\sphinxAtStartPar
\sphinxincludegraphics[height=80\sphinxpxdimen]{{k}.png}
&
\sphinxAtStartPar
\sphinxincludegraphics[height=80\sphinxpxdimen]{{ki}.png}
&
\sphinxAtStartPar
\sphinxincludegraphics[height=80\sphinxpxdimen]{{g}.png}
&
\sphinxAtStartPar
\sphinxincludegraphics[height=80\sphinxpxdimen]{{gi}.png}
&
\sphinxAtStartPar
\sphinxincludegraphics[height=80\sphinxpxdimen]{{p}.png}
&
\sphinxAtStartPar
\sphinxincludegraphics[height=80\sphinxpxdimen]{{pi}.png}
&
\sphinxAtStartPar
\sphinxincludegraphics[height=80\sphinxpxdimen]{{r}.png}
\\
\sphinxhline
\sphinxAtStartPar

&
\sphinxAtStartPar

&
\sphinxAtStartPar

&
\sphinxAtStartPar

&
\sphinxAtStartPar

&
\sphinxAtStartPar

&
\sphinxAtStartPar

\\
\sphinxbottomrule
\end{tabulary}
\sphinxtableafterendhook\par
\sphinxattableend\end{savenotes}

\sphinxstepscope


\chapter{Rozdział}
\label{\detokenize{zrzut/index:rozdzial}}\label{\detokenize{zrzut/index::doc}}

\section{Podrozdział}
\label{\detokenize{zrzut/index:podrozdzial}}
\begin{figure}[htbp]
\centering
\capstart

\noindent\sphinxincludegraphics[width=50\sphinxpxdimen]{{jezeli}.jpg}
\caption{jeżeli}\label{\detokenize{zrzut/index:id1}}\end{figure}


\begin{savenotes}\sphinxattablestart
\sphinxthistablewithglobalstyle
\centering
\begin{tabulary}{\linewidth}[t]{TTT}
\sphinxtoprule
\sphinxstyletheadfamily 
\sphinxAtStartPar
Tables
&\sphinxstyletheadfamily 
\sphinxAtStartPar
Are
&\sphinxstyletheadfamily 
\sphinxAtStartPar
Cool
\\
\sphinxmidrule
\sphinxtableatstartofbodyhook
\sphinxAtStartPar
col 1 is
&
\sphinxAtStartPar
left\sphinxhyphen{}aligned
&
\sphinxAtStartPar
\$1600
\\
\sphinxhline
\sphinxAtStartPar
col 2 is
&
\sphinxAtStartPar
centered
&
\sphinxAtStartPar
\$12
\\
\sphinxhline
\sphinxAtStartPar
col 3 is
&
\sphinxAtStartPar
right\sphinxhyphen{}aligned
&
\sphinxAtStartPar
\$1
\\
\sphinxbottomrule
\end{tabulary}
\sphinxtableafterendhook\par
\sphinxattableend\end{savenotes}


\subsection{Podpodrozdział}
\label{\detokenize{zrzut/index:podpodrozdzial}}
\sphinxAtStartPar
Jakiś \sphinxstylestrong{text}! I skreślone

\begin{sphinxadmonition}{note}{Uwaga!!!}

\sphinxAtStartPar
To jest adminition.$^{\text{1}}$
\end{sphinxadmonition}

\sphinxAtStartPar
\sphinxincludegraphics{{wtedy}.png}

\begin{sphinxadmonition}{tip}{Wskazówka:}
\sphinxAtStartPar
To jest wskazówka
\end{sphinxadmonition}

\sphinxAtStartPar
Pisanie \sphinxstyleemphasis{pisanie} \sphinxstylestrong{pisanie}


\chapter{Durga}
\label{\detokenize{zrzut/index:durga}}
\begin{sphinxadmonition}{note}{Wstawka niezdefiniowana}

\sphinxAtStartPar
Treść własnej wstawki (niezdefiniowanej).$^{\text{1}}$
\end{sphinxadmonition}

\sphinxAtStartPar
\sphinxincludegraphics{{wtedy}.jpg}

\sphinxAtStartPar
\sphinxincludegraphics{{wtedy1}.png}

\sphinxAtStartPar
Miód był \sphinxincludegraphics[width=50\sphinxpxdimen]{{kiedy}.jpg} wykorzystywany w celach kulinarnych co najmniej od mezolitu.

\sphinxAtStartPar
Jednakże inne \sphinxincludegraphics[width=50\sphinxpxdimen]{{kiedy}.jpg} produkty pszczele człowiek stosował do pielęgnacji, w sztuce oraz jako surowiec technologiczny prawdopodobnie od paleolitu. Najbardziej problematyczne jest to, że wszystkie te produkty nie pozostawiają po sobie żadnych śladów widocznych gołym okiem ani śladów uchwytnych mikroskopowo. Dopiero przy zastosowaniu metod lipidowych można uchwycić to, że pozostałości wyrobów pszczelich były wykorzystywane przez ludzi. Jeśli przechowywano je w naczyniach, które zachowały się do naszych czasów.

\begin{sphinxadmonition}{caution}{Ostrzeżenie:}
\sphinxAtStartPar
Uwaga!
\end{sphinxadmonition}


\begin{savenotes}\sphinxattablestart
\sphinxthistablewithglobalstyle
\centering
\begin{tabulary}{\linewidth}[t]{TTT}
\sphinxtoprule
\sphinxstyletheadfamily 
\sphinxAtStartPar
Tables
&\sphinxstyletheadfamily 
\sphinxAtStartPar
Are
&\sphinxstyletheadfamily 
\sphinxAtStartPar
Cool
\\
\sphinxmidrule
\sphinxtableatstartofbodyhook
\sphinxAtStartPar
col 1 is
&
\sphinxAtStartPar
left\sphinxhyphen{}aligned
&
\sphinxAtStartPar
\$1600
\\
\sphinxhline
\sphinxAtStartPar
col 2 is
&
\sphinxAtStartPar
centered
&
\sphinxAtStartPar
\$12
\\
\sphinxhline
\sphinxAtStartPar
col 3 is
&
\sphinxAtStartPar
right\sphinxhyphen{}aligned
&
\sphinxAtStartPar
\$1
\\
\sphinxbottomrule
\end{tabulary}
\sphinxtableafterendhook\par
\sphinxattableend\end{savenotes}

\begin{sphinxadmonition}{danger}{Niebezpieczeństwo:}
\sphinxAtStartPar
Nieprzespiecznie
\end{sphinxadmonition}

\sphinxAtStartPar
\sphinxincludegraphics[width=50\sphinxpxdimen]{{kiedy}.jpg} kiedy | \sphinxincludegraphics[width=50\sphinxpxdimen]{{jezeli}.jpg} jeżeli

\sphinxAtStartPar
\sphinxincludegraphics[width=50\sphinxpxdimen]{{kiedy}.jpg} kiedy \\
\sphinxincludegraphics[width=50\sphinxpxdimen]{{jezeli}.jpg} jeżeli

\sphinxAtStartPar
\sphinxincludegraphics[width=50\sphinxpxdimen]{{jezeli}.jpg} jeżeli
\sphinxincludegraphics[width=50\sphinxpxdimen]{{bedzie}.jpg} będzie
\sphinxincludegraphics[width=50\sphinxpxdimen]{{nie}.jpg} nie
\sphinxincludegraphics[width=50\sphinxpxdimen]{{kiedy}.jpg} kiedy
\sphinxincludegraphics[width=50\sphinxpxdimen]{{raz}.jpg} raz
\sphinxincludegraphics[width=50\sphinxpxdimen]{{zrzut/wtedy}.jpg} wtedy

\sphinxAtStartPar
\sphinxincludegraphics{{jezeli}.jpg} jeżeli
\sphinxincludegraphics{{bedzie}.jpg} będzie
\sphinxincludegraphics{{nie}.jpg} nie
\sphinxincludegraphics{{kiedy}.jpg} kiedy
\sphinxincludegraphics{{raz}.jpg} raz
\sphinxincludegraphics{{zrzut/wtedy}.jpg} wtedy

\sphinxAtStartPar
\sphinxincludegraphics{{jezeli}.jpg}
\sphinxincludegraphics{{bedzie}.jpg}
\sphinxincludegraphics{{nie}.jpg}
\sphinxincludegraphics{{kiedy}.jpg}
\sphinxincludegraphics{{raz}.jpg}
\sphinxincludegraphics{{zrzut/wtedy}.jpg}


\section{Durga podrozdział}
\label{\detokenize{zrzut/index:durga-podrozdzial}}
\begin{sphinxadmonition}{error}{Błąd:}
\sphinxAtStartPar
Błąd
\end{sphinxadmonition}


\subsection{Durga podpodrozdział}
\label{\detokenize{zrzut/index:durga-podpodrozdzial}}
\begin{sphinxadmonition}{important}{Ważne:}
\sphinxAtStartPar
Importante
\sphinxincludegraphics[width=100\sphinxpxdimen]{{hha}.jpg}
\end{sphinxadmonition}

\begin{sphinxadmonition}{note}{Informacja:}
\sphinxAtStartPar
Note
\end{sphinxadmonition}


\begin{sphinxseealso}{Zobacz także:}

\sphinxAtStartPar
Patrz też seealso


\end{sphinxseealso}


\begin{sphinxadmonition}{warning}{Ostrzeżenie:}
\sphinxAtStartPar
Warning
\end{sphinxadmonition}


\subsubsection{Durga podpodpodrozdział}
\label{\detokenize{zrzut/index:durga-podpodpodrozdzial}}
\sphinxAtStartPar
\sphinxincludegraphics{{hha}.jpg}

\sphinxAtStartPar
Na podstawie tego typu badań jesteśmy w stanie uchwycić występowanie materiału dotychczas uznawanego za nieuchwytny archeologicznie. Dodatkowo większość przedmiotów związanych z tradycyjnym pszczelarstwem w Europie także wykonywano z surowców organicznych, które rzadko zachowują się w kontekście archeologicznym. Należą do nich choćby ule, w tym tradycyjne dla niektórych rejonów zachodniej Europy skeps, czyli plecione ule, wykonywane najczęściej z wikliny.


\chapter{Odsyłacze robocze}
\label{\detokenize{index:odsylacze-robocze}}\begin{itemize}
\item {} 
\sphinxAtStartPar
\sphinxhref{https://bashtage.github.io/sphinx-material/rst-cheatsheet/rst-cheatsheet.html}{RST Cheatsheet}

\item {} 
\sphinxAtStartPar
\sphinxhref{https://myst-parser.readthedocs.io/en/latest/index.html}{MYST docs}

\end{itemize}



\renewcommand{\indexname}{Indeks}
\printindex
\end{document}